\documentclass{article}
\usepackage{listings}
\lstset{
    language=Prolog,
    breaklines=true
}
\usepackage{hyperref}

\begin{document}
    {
        \centering
        
        \textbf{\Huge{Codeable}}
        
        \vspace*{1em}
        \Large{\emph{A highly-readable programming language}}
        
        \Large{\emph{designed for software newcomers.}}
        
        \vspace*{1em}
        \href{https://github.com/jhale1805/SER502-Spring2022-Team15}{github.com/jhale1805/SER502-Spring2022-Team15}

        \vspace*{5em}
        \Large{Created by}
        
        \vspace*{1em}
        \normalsize{Rithvik Arun}
        
        Joseph Hale 
        
        Jacob Hreshchyshyn 
        
        Jacob Janes 
        
        Sai Nishanth Vaka
        
        \vspace*{5em}
        \Large{In partial fulfillment of the requirements for}
        
        \vspace*{1em}
        \normalsize{SER502}
        
        Programming Languages and Paradigms
        
        by Dr. Ajay Bansal
        
        \vspace*{1em}
        in the 
        
        Spring 2022 Semester
        
        \vspace*{1em}
        at

        Arizona State University

    }
    
    %%%%%%%%%%%%%%%%%%%%%%%%%%%%%%%%%%%%%%%%%%%%%%
    %%% Grammar
    \newpage
    \section{Grammar}
    \lstinputlisting{grammar.md}
    
    %%%%%%%%%%%%%%%%%%%%%%%%%%%%%%%%%%%%%%%%%%%%%%
    %%% Implementation Tooling
    \newpage
    \section{Implementation Tooling}
    
    \subsection{Tokenizer}

    In our tokenizer, we will generate a tokenized string for each line
    of Codeable code, and will do this by utilizing our Javascript
    front end to pull each line as a new line of Codeable text. After
    generating the tokenized text, we will send all lines to our Prolog
    DCG rule interpreter for parsing and returning a tree for our
    subtasks.

    \subsection{Parser}

    For our parser, we will be utilizing Prolog and tokenized text in
    order to generate parse trees. As such, Prolog will be used as our
    interpreter, and this will be done with a set of Definite Clause
    Grammar rules and a Recursive Descent Parser. In order to make this
    as easy as possible, we are going to utilize Prolog rules to
    extract the grammar rules we have laid out, and provide quick and
    accurate trees that represent the functions provided by a user of
    Codeable.

    \subsection{Evaluator}

    After receiving a parse tree from Prolog, our system shall evaluate
    the parse tree and perform the calculations on those interactions
    based on what the user has provided in the Codeable language, or a
    false response if the user form is not valid or cannot be
    evaluated. In this approach, we can simplify the evaluation process
    as we already have the rules in Prolog, and will reduce the
    complexity of parsing a parse tree with limited context of the
    grammar rules.

    \subsection{Runtime}

    Writing programs in Codeable and running them can occur anywhere
    that a Prolog installation is available. In line with our hopes to
    make this language as easy to use as possible for software
    newcomers, we will create a website-based playground for writing
    Codeable programs.
    
    To accomplish this, we will leverage the open-source JavaScript
    library \href{https://github.com/tau-Prolog/tau-prolog}{Tau-Prolog}
    which enables parsing and evaluating Prolog code in the browser. We
    will have a text box for writing the Codeable instructions and an
    output window. Behind the scenes, the Prolog rules that constitute
    the Tokenizer, Parser, and Evaluator described previously will be
    fed into Tau-Prolog to process the user's Codeable instructions and
    return the output to be rendered in the webpage.

    \subsection{Data Structures}
    
    \subsubsection{Array}
    
    One of the data structures that will be used by the
    parser/interpreter will be arrays. Arrays will be useful to group a
    fixed amount of data and run operations on that data. Arrays will
    follow common structure with indices starting at 0, and supporting
    only one data type at a time.

    \subsubsection{Lists}
    
    Another data structures that will be used by the parser/interpreter
    will be lists. Lists will be useful for ordering dynamic data. In
    other words, if the data does not have a fixed amount, lists will
    provide an easy way to add and remove data on the fly. The list
    will follow a common structure with functionality such as GET,
    REMOVE, CONTAINS, REMOVE\_BY\_VALUE, and ADD.

    \subsubsection{Trees}
    
    The last data structure that will be used by the parser/interpreter
    will be trees. Trees will be useful to group data that have
    hierarchical relationships. With trees we can also run many
    traversals such as BFS and DFS to search for data that we might
    want. Trees will follow a binary pattern where each parent node can
    have 0,1 or 2 child nodes.

    
    %%%%%%%%%%%%%%%%%%%%%%%%%%%%%%%%%%%%%%%%%%%%%%
    %%% Contribution Plan
    \newpage
    \section{Contribution Plan}

    \begin{table}[h]
        \begin{tabular}{| l | l |}
            \hline
            Task & Assigned To \\
            \hline
            Encode Boolean values in runtime environment & Sai V \\
            Encode numeric type in runtime environment & Sai V \\
            Encode string support in runtime environment & Rithvik \\
            Encode assignment support in runtime environment & Sai V \\
            Encode ternary operator in runtime environment & Jacob J \\
            Encode if-then-else construct in runtime environment & Jacob J \\
            Encode for loop in runtime environment & Jacob H \\
            Encode while loop in runtime environment & Jacob H \\
            Encode syntactic sugar representation of for loop (e.g. for i in range(2, 5)) & Jacob H \\
            Encode print construct & Rithvik \\
            Set up runtime environment & Joseph \\
            Encode data structures in runtime environment (if time allows) & Jacob J \\
            Set up interpreter prolog rules to feed into Tau prolog & Jacob J \\
            Set up compiler/interpreter environment with Tau & Joseph \\
            Create PDF of language presentation & Joseph \\
            Create and upload YouTube video of language presentation. & Rithvik \\
            \hline
        \end{tabular}
    \end{table}

    %%%%%%%%%%%%%%%%%%%%%%%%%%%%%%%%%%%%%%%%%%%%%%
    %%% Example Programs
    \newpage
    \section{Example Programs}
    
    \subsection{Quadratic Formula}
    \lstinputlisting{../data/quadratic_formula.md}
    
    \subsection{Factorial}
    \lstinputlisting{../data/factorial.md}
    
    \subsection{Towers of Hanoi}
    \lstinputlisting{../data/tower-of-hanoi.md}
    
    
\end{document}